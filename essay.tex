\documentclass[12pt]{article}
\usepackage[english]{babel}
\usepackage[utf8]{inputenc}
\usepackage[a4paper, total={6in, 8in}]{geometry}
\usepackage{amsmath}
\usepackage{graphicx}
\usepackage{hyperref}
\graphicspath{example/path} 
\raggedright

\date{}
\author{Marking Code: Z0171688}
\title{
Research Project 2022: How Do Primitive Methodologies Of Migrant Studies Compare To The Novel Methods 
Presented In "Who Ties The World Together?" (Chi, G. et al, 2020)
}

\begin{document}
\parindent=20pt

\maketitle

\section{Introduction}

The size and the nature of the connections that are formed between countries is often the 
basis for their trade, levels of peace, and their national and cultural identity, (Hollis, M., 1990). 
The ways in which states are connected, through trade and politics at the macro level is 
well understood and widely written about, but these studies have relied 
on data which is purely macro, such as customs data, foreign investments, and trade 
data, (Gandolfo, G. 1998), newer studies have been able to take into account online data 
as a novel data source, but this has been limited in scope, and did not extend to the
true potential of this sort of data, (State, B. 2015)(Garcia-Gavilanes).
The potential of this method of data was better demonstrated by "Who Ties the World Together?", (Chi, G. 2020).
The researchers in this paper were given access to the Meta dataset, and were therefore able to 
analyse relationships in unprecedented detail and frankness. Their paper demonstrates the degree
to which researchers have not been able to utilise micro connections in their studies - as they 
put it "\textit{who} connects the world", (p 1, Introduction). 

Understanding International Relations in this way facilitates new methods of evaluating the 
theories that constitute it. For instance, Hafner-Burton et al. (2009) aims to better 
understand networks as the relations between individual agents, what "define[s], enable[s] and 
constrain[s]" them, but feels limited by "trivial conclusions" and "unproven assertions". 
The research demonstrated by Chi, G et al (2020) makes the available assertions more accurate
and the conclusions more interesting. Coleman (1994) discusses the disconnect in the study  
of macro versus micro, as data for each is unable to cleanly overlap.

The results of Chi, et al (2020) render previous research more open to criticism for the 
uncoupling of micro and macro, and the lack of accounting for the role of individual in 
the international relationships between countries. They demonstrate that migrants are a 
binding position in social networks, bridging the gaps between communities, and that by 
reshaping society at the domestic level, they have a substantial effect at the international 
level. This is clear when you consider the branching effect: when an individual moves 
to a new country, the ties that they make in the host country and connected back to the ties 
that they have in their home country. Their results are consistent with other research 
that uses more primitive methods, such as Perkins et al (2013), but as will be the main conclusion of this project: although the final 
results/conclusions are often similar when researchers use primitive versus novel datasets, 
the accuracy of the results and therefore the weight that can be 
given to the conclusions when modern datasets are utilised is vital to take into account 
when considering the viability of older methods in the face of the accuracy and depth of 
the social media companies' datasets.
An example of their accuracy is the ability to graphically model the effects that migrants 
play in a community, using Facebook "friending" as edge between individuals as nodes. As such,
the researchers are able to simulate the growth patterns of the same communities, but with the 
migrants removed: this is possible as the Facebook dataset has existed for years, and so 
the same trends that took place can be reproduced, while editing certain variables. Using a 
less accurate method of study, such a simulation would exacerbate the inherent inaccuracies, 
due to simple statistics.

Considering the points discussed so far, it would be a useful contribution to the literature to 
analyse how this novel style of research compares to more primitive methods: Meta has so far 
only released two papers using their dataset for this specific study (the study of migration 
and the relationships between migrants) from what I can find, as such, it is evidently a 
nascent area of study. This project will primarily compare the Chi et al. (2020) paper to 
existing literature, as of the two papers available, it is best suited to international 
analysis. As the research available using this dataset is currently so limited, I have no option 
but to focus on a single paper. However, this is not restrictive of the project: the paper is 
a clear demonstration of the effectiveness, the scope, the detail, and the frankness of this 
nascent methodology (using social media datasets), hence this projected should not be disregarded 
for its lack of breadth. This is a deeply interesting area to analyse.

\section{Existing Literature}

\textit{
	This section will look at existing literature exemplifying some of the existing 
	methodologies which are either most prevalent, or show commonalities with the methodology 
	of Chi et al. (2020), such as literature studying networks. Unfortunately, as this paper is so recent, I was unable to find 
	literature pertaining exactly to this question (a critique of primitive methods in the 
	face of "Who Ties The World Together?", (2020)). I consider this a positive aspect, as 
	it demonstrates the contritbutions of this project. However, it does mean that this 
	section will be briefly reviewing and laying out the literature that will be more closely
	considered in the analysis section.
}
\newline

As stated in the introduction, existing studies have demonstrated somewhat of a disconnect 
between macro and micro, as their methodologies would lose the individuals in order to 
study flows of people or to gain a broad understanding of wide network. One such example 
is Amin, et al. (1994), which demonstrates that in the advent of globalisation, certain 
"global cities", (Short, et al. 1996), can be well analysed as part of an international 
network, in how they: facilitate trade through offering global business hubs; this flow 
of money attracts travel, creating diverse communities, and one can then assert that creates 
further business through entrepreneurs filling gaps in markets with inspiration from their 
home country, for instance foreign restaurants, (these networks comprised of individuals are shown by Chi et al, (2020) to 
be vital to international networks). 

However, other 
papers have criticised some of the assertions made by researchers focusing on this subject, 
such as Derudder, (2006). Derudder criticises the lack of empirical evidence brought by 
in these analyses: despite their sensible nature, a dearth of empirical evidence is 
a frustration to the argument, (this was of course a valid criticism before the Chi et al 
paper, which \textit{was} able to give empirical evidence). Short et al, (1996), although 
welcome to the notion of urban networks and their global impact, has the primary goal 
of discussing and demonstrating this 'dirty little secret'. Its main criticism is that 
there needs to be a greater availability of comparative statistics, as the researchers 
consider other studies to be repeating common assertions without using data which open 
and accessible. In the wake of this, newer studies sought tackle this dearth of data 
specifically, such as Mahutga, et al.(2010). While these efforts have served to create 
greater data availability in order to study urban networks, these studies still suffer 
from the previously mentioned issues where individuals are lost at the macro level. 
Taking Mahutga (2010), the researchers use airline data as their empirical evidence 
for grading urban hubs and their internationality. But as discussed, this method 
is unable to analyse relationships at the individual level: expressing this graphically,
 one can only place nodes 
modeling an inherently macro network, comprised of edges indicating flows of thousands 
of people, all deeply detached from any micro analysis. This is very limiting of 
any network-oriented.

Malecki (2002) and Townsend (2001) both used internet connections between urban hubs 
in order to analyse their position within a global network, measuring speed and traffic
in order to make assertions about them. This suffers the same detachment from micro 
studies that airline data suffers while also being potentially misleading, as 
less important cities in the network may act as "gateways" (Chi et al, 2020) to
larger hubs, so the data would over state their position, (Rutherford, J. et al, 2004).

Other attempts have been made with more modern datasets to maintain more granular 
information. Takhteyev (2012) used Twitter data to analyse social ties. However, this 
research was done in 2012, when social media (especially Twitter) was practically 
nascent, only being regularly used by a very small proportion of the population 
relative to today. Furthermore, it analysed what were the predictors of Twitter ties, 
for instance regularity of flights between areas and then comparing this to the 
numbers of Twitter ties between these same areas. This seems much more a study 
of Twitter itself, than an analysis of migrant networks using the incredible data that 
social media makes available - case in point, at this time, this data hardly existed, 
relative to its size today. Other studies were similarly conducted, marvelling more 
at the novelty of social media, and using other metrics to study its linking effect, 
rather than primarily using the links on display to make assertions and conclusions. 
Such examples are Backstrom (2012), which uses Facebook data rather than Twitter data: 
the paper notes the order of magnitude greater sample that they have available compared 
to Milgram (1967), a study with a similar research topic, analysing networks of individuals 
and how they are distanced from one another. Reading Chi et al (2020), the same comparison 
of orders of magnitude larger sample size and granularity is blatant, as the popularity 
of social media and its penetration into society has exploded. Furthermore, a particularly  
important aspect to consider is the popularity of social media abroad: at the time of this 
research Facebook revenue outside of the United States and Canada was just under three
million dollars, now it is over 25 million (Dixon, S. 2023). This demonstrates the ineffectiveness of social 
media for measuring migrant networks at the time of this research versus today.

Some earlier studies of human mobility and migrants' position in networks used mobile phone data to 
track movement, (Gonzalez, et al. 2008), or communication by analysing phone traffic (Perkins et al, 2013).
This is proves more effective than using other forms of proxy data mentioned earlier,  
however it is still less accurate, and uses a smaller cohort than is possible with social media 
data, (Gonzalez used 100,000 callers, which pales at the millions used by Chi (2020)). 
Kikas et al, (2015), served as a proof of concept of the effectiveness of using social 
media data to analyse and predict migration patterns, noting the usefulness for tracking 
demographic changes. The reasearchers were able to use login data from 15 million 
Skype users, demonstrating the enormous size of the datasets available when using social 
networking data. However, this method used the IP addresses from which users were logging 
on in order to understand migration. This method is only accurate at the country level (Kikas, section 2.1)
and therefore can only used to see the migration of individuals between countries,
sort of as an upgraded usage of the flight data mentioned earlier. No modelling or 
analysis can be done of where migrants go once they are in a country, or who they interact 
with, neutering any ability to assess global hubs or migrants impact on the social 
networks in the host country, or how their integration into a host country impacts their 
home country.

Herdagdelen et al (2016) is perhaps the most interesting research and is the most 
pertinent to this project, as it is sort of a precursor to the Chi et al paper. The researchers 
note that their study is "the first large-scale analysis of immigrants’ social networks in the United States" (page 83, Conclusion),
and offers similar insight into the relationships that migrants form in host countries, such 
as a predisposition to form ties with compatriots. They do note some inaccuracies however, 
such as their usage of users' home countries as they are reported on their home page. This 
removes any user that does not report their home page from the data set. It also biases 
the dataset towards those who are perhaps proud of their origins, as they are more likely 
to display their country of origin, (there may, of course be other reasons for not reporting 
origin, this is just on speculation by the researchers). They note that issues like this, 
as well as the potential for online ties to not wholly reflect offline ties, mean that 
their methods require greater testing. The method obviously should be lauded though: as 
the researchers state, one of their motivations was the inability to gain micro level insight
from existing research methods, such as surveys, therefore warranting the use of more 
personal data. 


Finally, Zagheni et al (2017) uses the Facebook advertising API to measure migrant numbers 
in America. Their study is primarily to measure stocks and demographics of migrants, 
rather than specifically network bases motivation, however their methodology is similar to 
that of Chi (2020) and Herdagdelen (2016), in that it is able to survey a very large portion of 
the population through social media companies' datasets. They also note biases in their data 
however, as they find that their statics slightly overestimate numbers of younger 
migrants, and underestimate the numbers of older migrants. This is evidently because a 
greater proportion of social media users are young. Furthermore, their use of the advertiser 
API, although novel, is limited to simply numbers, and again cannot go beyond measuring stocks. 
Furthermore, it may be unclear if the API is completely candid in its measures of stocks, 
as it may be the case that Facebook is more likely to push advertisements to a 
specific demographic, due to the high numbers of younger users, therefore reducing the 
exposure of older users to the sample.

Reviewing the existing literature, I have included research which either demonstrates more 
primitive methods of data collection pertinent to this subject, or research which 
utilises social media or other online-oriented datasets but does not make use of the 
same level of potential as does "Who Ties The World Together?", (Chi et al, 2020). These 
inclusions are relevant for this project as it is important to gain a perspective on 
what research methodology used to look like, how it has progressed, and where it is currently,
in order to answer my research question.

\section{Methodology}

My methodology will be to focus in on particular papers mentioned in the previous section. 
I will analyse their research and their conclusions, in order to compare them to the 
detail and the accuracy of the Chi et al (2020) research paper. In this way, I will be 
able to deduce what the exact limitations are of previous research methods and how they 
effect the outcomes of the studies. 

I will not only be able to see what detracts from previous studies, but also understand 
where the Chi et al (2020) paper falls short, for instance, are the biases of Zagheni 
et al (2017) also present, or does using friend status and relationship data change this.
I will also discuss whether previous studies could have been improved with this novel 
data, or whether the questions of the research were themselves limited by the expectations
of the researchers, and what they knew was possible with the data that was available. Could 
these questions have been revised faced with the enormous, and detailed Meta (Facebook) dataset?

\section{Results and Analysis}

\subsection{The Cons Of The Novel Dataset and Methodology}
I will begin by evaluating the limitations of the methodology used in Chi et al (2020). The 
primary limitation of the dataset is that it is not wholly representative of the entire 
population, rather it represents a biased subset of the population: Mellon et al, (2017)
analysed Twitter and Facebook users in the United Kingdom and compared various demographics 
and their political views. The results of the paper were that the user base of social 
media platforms is generally younger, better educated, and more politically active 
than the population average. This is of course to be expected, and seems a sensible conclusion, 
but it is something of an indictment against the usefulness of the data for specific studies, (
	this fact will be useful later
), and these results are reinforced by other studies, such as Greenwood et al (2016), and 
Duggan (2015). Furthermore, social media users are more likely to be motivated by and interested 
in politics (after controlling for demographics in the case of Facebook, (Mellon et al, 2017)),
while also being less likely to vote, but when they do vote, they will most likely vote to the 
left, and for a more liberal agenda. These facts are of course important to keep in mind 
when doing any sort of social science study, but they are more pertinent for specific kinds 
of studies: a paper which is specifically trying to gauge the political motivations of an 
entire country cannot utilise the social media dataset without being careful to control 
for bias, and taking other steps to ensure that the study is accurate. But there are myriad
other research which would not need to be so careful. In Chi et al (2020), the motivation 
is to study social networks of individuals, and what effects migrants have when they integrate 
into a host country. Such a study finds much greater benefit in being able to use an 
incredibly large sample size, which covers a global scope, with detailed analytics on 
individual relationships and social ties, than these drawbacks are to its detriment. 

Perkins et al (2013) does not seem to suffer the same bias drawbacks that Chi et al (2020) does, 
and yet they found similar results (that migrants are a vital part of communication 
networks). Either, Perkins suffers the same biases and therefore the same groups are not 
being represented by the sample, or Perkins is void of the bias issues, is representative, 
and yet still comes to the same conclusions. Both these outcomes indicates that the social 
media dataset is useful in analysing global social networks at a micro and macro level: 
each study represents one of the primary methods of available communication, therefore 
any migrant not using them will not be contributing to the bridging effect present in 
both studies, and therefore is unimportant to analysing global social networks, as they are 
not a part of any substantial network. Of course it would be an interesting study in 
itself to analyse those migrants who exist in social groups not captured by the primary 
methods of communication, but they are not necessarily an important section to miss in the 
type of study being presently discussed. If migrants were found to not have a bridging 
effect, as they were not utilising communication networks, the social media dataset would 
be unhelpful, as migrants are not being represented, but as it stands, an important 
proportion of them are, and they are useful to a study of global social networks. This 
argument would demonstrate that while usage of this novel dataset does need to be 
tempered and controlled, for specific studies, the benefits greatly outweigh detriment. 

The second criticism is the availability of the dataset...

\subsection{The Benefits of the Dataset}

The great advantages of the Facebook dataset used by Chi et al (2020) are its size, 
its detail, and its frankness. 

First its size. The dataset is enormous compared to other, even similar style datasets: 
for instance, take the Skype dataset used in Kikas et al (2015). Altough the researchers 
were using data from a socially oriented online network, they only had access to 100,000 
users, which pales in comparison the billions available to Chi et al. There are 
certainly, statistically, diminishing returns in accuracy as a dataset grows, but billions 
compared to even millions is a substantial upgrade in accuracy and breadth of scope. 
Especially as the data used by Kikas is login data from between 2007 and 2011. At the time, 
wifi speeds, camera quality, and microphone quality necessary for video call (the primary 
usage of Skype, particularly at the time, although the app does offer messaging) would
suggest that the individuals surveyed would have a greater bias towards more well off 
migrants. Of course a similar bias would exist to an extent in the Facebook dataset, 
however to a much smaller degree, as all that is necessary is a smartphone, the price of 
which has fallen substantially (a large reason for Facebook and other social networking 
sites and apps can have such a large user base, even in countries with lower average incomes.)
The size of such a dataset better enables the modelling seen in Chi et al, (2020). With such 
a dataset, Kikas could have more accurately plotted the connections between countries (the 
primary focus of the paper) as they would have had a much larger sample of the global 
population to work with. As the researchers were studying cross-country migration, 
100,000 points actually seems relatively small: the number of migrants worldwide in June 2019
was estimated at almost 272 million, (Bauloz et al , World Migration Report 2020). As such, 
to only be able to look at the logins of less than 0.04\% of these is underwhelming, 
and certainly does little to properly comprehend cross country ties. The Facebook dataset 
however, accounts for close to half of the world's population, and shows a smaller degree 
of bias (although there is still important bias that should be accounted for). Using this 
dataset, Chi et al (2020) were able to measure the proportion of international ties that 
involve migrants (17.09\%, Results subsection 4.1). With only a measure of 100,000 
individuals (and data of login locations by country, with basically no data on country 
of origin), Kikas et al are completely unable to come to any similar conclusion (at least 
while ensuring accuracy). They could even measure the proportion of these international 
relationships which are constituted by migrant to local ties, and migrant to country of origin 
ties, (39.4\% and 27.88\% respectively). One would assume that in trying to measure cross-country 
connections, Kikas et al would have naturally made these same measurements, were the data 
available. 

\section{Conclusion}

\pagebreak 
\section{Bibliography}

Bauloz, C., Vathi, Z. and Acosta, D., 2019. Migration, inclusion and social cohesion: Challenges, recent developments and opportunities. World Migration Report 2020, pp.186-206.

Greenwood S, Perrin A and Duggan M (2016) Social Media Update
2016. Pew Research Center. Available at: http://assets.pewre-
search.org/wp-content/uploads/sites/14/2016/11/10132827/
PI_2016.11.11_Social-Media-Update_FINAL.pdf.

Duggan M (2015) Mobile messaging and social media 2015. Pew
Research Center. Available at: http://www.pewinternet.org/
files/2015/08/Social-Media-Update-2015-FINAL2.pdf.

Mellon, J., Prosser, C.: Twitter and Facebook are not representative of the general
population: Political attitudes and demographics of British social media users. Res.
Polit. 4(3) (2017)

Perkins, R., Neumayer, E.: The ties that bind: the role of migrants in the uneven
geography of international telephone traffic. Global Netw. 13(1), 79–100 (2013)

Gonzalez, M.C., Hidalgo, C.A. and Barabasi, A.L., 2008. Understanding individual human mobility patterns. nature, 453(7196), pp.779-782.

S. Dixon and 13, F. (2023) Facebook: Average revenue per user region 2022, Statista. Available at: https://www.statista.com/statistics/251328/facebooks-average-revenue-per-user-by-region/ (Accessed: April 29, 2023). 

Zagheni, E., Weber, I., Gummadi, K.: Leveraging Facebook’s Advertising Platform
to Monitor Stocks of Migrants. Popul. Dev. Rev. 43(4), 721–734 (2017)

Herdagdelen, A., State, B., Adamic, L., Mason, W.: The social ties of immigrant
communities in the United States. In: Proc. WebSci’16. pp. 78–84 (2016)

Kikas, R., Dumas, M., Saabas, A.: Explaining International Migration in the Skype
Network: The Role of Social Network Features. In: Proceedings of the 1st ACM
Workshop on Social Media World Sensors. pp. 17–22 (2015)

Backstrom, L., Boldi, P., Rosa, M., Ugander, J. and Vigna, S., 2012, June. Four degrees of separation. In Proceedings of the 4th Annual ACM Web Science Conference (pp. 33-42).

Milgram, S., 1967. The small world problem. Psychology today, 2(1), pp.60-67.

Takhteyev, Y., Gruzd, A., Wellman, B.: Geography of Twitter networks. Soc. Netw.
34(1), 73–81 (2012)

Rutherford, J., Gillespie, A., Richardson, R.: The territoriality of Pan-European
telecommunications backbone networks. J. Urban Technol. 11(3), 1–34 (2004)

Malecki, E.J.: The Economic Geography of the Internet’s Infrastructure. Econ.
Geogr. 78(4), 399–424 (2002)

Townsend, A.M.: Network Cities and the Global Structure of the Internet. Am.
Behav. Sci. 44(10), 1697–1716 (2001)

Mahutga, M.C., Ma, X., Smith, D.A. and Timberlake, M., 2010. Economic globalisation and the structure of the world city system: the case of airline passenger data. Urban Studies, 47(9), pp.1925-1947.

Derudder, B. (2006). On Conceptual Confusion in Empirical Analyses of a Transnational Urban Network. Urban Studies, 43(11), 2027–2046. https://doi.org/10.1080/00420980600897842

Amin, A. and Thrift, N.J., 1994. Living in the global. In Globalisation, institutions and regional development in Europe (pp. 1-22). Oxford University Press.

Short, J.R., Kim, Y., Kuus, M., Wells, H.: The Dirty Little Secret of World Cities
Research: Data Problems in Comparative Analysis. Int. J. Urban Reg. Res. 20(4),
697–717 (1996)

Derudder, B., Witlox, F., Taylor, P.J.: U.S. Cities in the World City Network.
Urban Geogr. 28(1), 74–91 (2007)

Derudder, B., Witlox, F.: An Appraisal of the Use of Airline Data in Assessing the
World City Network: A Research Note on Data. Urban Stud. 42(13), 2371–2388
(2005)

Andris C.,Cavallo S.,Dzwonczyk E.,Clemente-Harding L.,Hultquist C. and Ozanne M. (2019) Mapping the Distribution and Spread of Social Ties Over Time: A Case Study Using Facebook Friends. Connections, Vol.39 (Issue 1), pp. 1-17. https://doi.org/10.21307/connections-2019-007

Lex Fridman, 26 Feb 2022, Interview With Mark Zuckerberg, link: https://www.youtube.com/watch?v=5zOHSysMmH0

Perkins, R., Neumayer, E.: The ties that bind: the role of migrants in the uneven
geography of international telephone traffic. Global Netw. 13(1), 79–100 (2013)

Coleman, J.S., 1994. Foundations of social theory. Harvard university press.

Hafner-Burton, E.M., Kahler, M., Montgomery, A.H.: Network analysis for inter-
national relations. Int. O. 63(3), 559–592 (2009)

State, B., Park, P., Weber, I., Macy, M.: The Mesh of Civilizations in the Global
Network of Digital Communication. PLOS ONE 10(5), e0122543 (2015)

Garcia-Gavilanes, R., Mejova, Y., Quercia, D.: Twitter ain’t without frontiers:
economic, social, and cultural boundaries in international communication. In: Proc.
ICWSM’14. pp. 1511–1522 (2014)

Leskovec, J., Horvitz, E.: Planetary-scale views on a large instant-messaging net-
work. In: Proceeding of the 17th international conference on World Wide Web -
WWW ’08. p. 915 (2008)

Chi, G., State, B., Blumenstock, J.E. and Adamic, L., 2020. Who Ties the World Together? Evidence from a Large Online Social Network. In Complex Networks and Their Applications VIII: Volume 2 Proceedings of the Eighth International Conference on Complex Networks and Their Applications COMPLEX NETWORKS 2019 8 (pp. 451-465). Springer International Publishing.

Gandolfo, G. and Gandolfo, G., 1998. International trade theory and policy (pp. 233-234). Berlin: Springer.

Hollis, M., Smith, S.: Explaining and Understanding International Relations.
Clarendon Press (1990)

\end{document}
